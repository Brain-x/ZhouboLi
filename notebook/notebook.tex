
% Default to the notebook output style

    


% Inherit from the specified cell style.




    
\documentclass[11pt]{article}

    
    
    \usepackage[T1]{fontenc}
    % Nicer default font (+ math font) than Computer Modern for most use cases
    \usepackage{mathpazo}

    % Basic figure setup, for now with no caption control since it's done
    % automatically by Pandoc (which extracts ![](path) syntax from Markdown).
    \usepackage{graphicx}
    % We will generate all images so they have a width \maxwidth. This means
    % that they will get their normal width if they fit onto the page, but
    % are scaled down if they would overflow the margins.
    \makeatletter
    \def\maxwidth{\ifdim\Gin@nat@width>\linewidth\linewidth
    \else\Gin@nat@width\fi}
    \makeatother
    \let\Oldincludegraphics\includegraphics
    % Set max figure width to be 80% of text width, for now hardcoded.
    \renewcommand{\includegraphics}[1]{\Oldincludegraphics[width=.8\maxwidth]{#1}}
    % Ensure that by default, figures have no caption (until we provide a
    % proper Figure object with a Caption API and a way to capture that
    % in the conversion process - todo).
    \usepackage{caption}
    \DeclareCaptionLabelFormat{nolabel}{}
    \captionsetup{labelformat=nolabel}

    \usepackage{adjustbox} % Used to constrain images to a maximum size 
    \usepackage{xcolor} % Allow colors to be defined
    \usepackage{enumerate} % Needed for markdown enumerations to work
    \usepackage{geometry} % Used to adjust the document margins
    \usepackage{amsmath} % Equations
    \usepackage{amssymb} % Equations
    \usepackage{textcomp} % defines textquotesingle
    % Hack from http://tex.stackexchange.com/a/47451/13684:
    \AtBeginDocument{%
        \def\PYZsq{\textquotesingle}% Upright quotes in Pygmentized code
    }
    \usepackage{upquote} % Upright quotes for verbatim code
    \usepackage{eurosym} % defines \euro
    \usepackage[mathletters]{ucs} % Extended unicode (utf-8) support
    \usepackage[utf8x]{inputenc} % Allow utf-8 characters in the tex document
    \usepackage{fancyvrb} % verbatim replacement that allows latex
    \usepackage{grffile} % extends the file name processing of package graphics 
                         % to support a larger range 
    % The hyperref package gives us a pdf with properly built
    % internal navigation ('pdf bookmarks' for the table of contents,
    % internal cross-reference links, web links for URLs, etc.)
    \usepackage{hyperref}
    \usepackage{longtable} % longtable support required by pandoc >1.10
    \usepackage{booktabs}  % table support for pandoc > 1.12.2
    \usepackage[inline]{enumitem} % IRkernel/repr support (it uses the enumerate* environment)
    \usepackage[normalem]{ulem} % ulem is needed to support strikethroughs (\sout)
                                % normalem makes italics be italics, not underlines
    

    
    
    % Colors for the hyperref package
    \definecolor{urlcolor}{rgb}{0,.145,.698}
    \definecolor{linkcolor}{rgb}{.71,0.21,0.01}
    \definecolor{citecolor}{rgb}{.12,.54,.11}

    % ANSI colors
    \definecolor{ansi-black}{HTML}{3E424D}
    \definecolor{ansi-black-intense}{HTML}{282C36}
    \definecolor{ansi-red}{HTML}{E75C58}
    \definecolor{ansi-red-intense}{HTML}{B22B31}
    \definecolor{ansi-green}{HTML}{00A250}
    \definecolor{ansi-green-intense}{HTML}{007427}
    \definecolor{ansi-yellow}{HTML}{DDB62B}
    \definecolor{ansi-yellow-intense}{HTML}{B27D12}
    \definecolor{ansi-blue}{HTML}{208FFB}
    \definecolor{ansi-blue-intense}{HTML}{0065CA}
    \definecolor{ansi-magenta}{HTML}{D160C4}
    \definecolor{ansi-magenta-intense}{HTML}{A03196}
    \definecolor{ansi-cyan}{HTML}{60C6C8}
    \definecolor{ansi-cyan-intense}{HTML}{258F8F}
    \definecolor{ansi-white}{HTML}{C5C1B4}
    \definecolor{ansi-white-intense}{HTML}{A1A6B2}

    % commands and environments needed by pandoc snippets
    % extracted from the output of `pandoc -s`
    \providecommand{\tightlist}{%
      \setlength{\itemsep}{0pt}\setlength{\parskip}{0pt}}
    \DefineVerbatimEnvironment{Highlighting}{Verbatim}{commandchars=\\\{\}}
    % Add ',fontsize=\small' for more characters per line
    \newenvironment{Shaded}{}{}
    \newcommand{\KeywordTok}[1]{\textcolor[rgb]{0.00,0.44,0.13}{\textbf{{#1}}}}
    \newcommand{\DataTypeTok}[1]{\textcolor[rgb]{0.56,0.13,0.00}{{#1}}}
    \newcommand{\DecValTok}[1]{\textcolor[rgb]{0.25,0.63,0.44}{{#1}}}
    \newcommand{\BaseNTok}[1]{\textcolor[rgb]{0.25,0.63,0.44}{{#1}}}
    \newcommand{\FloatTok}[1]{\textcolor[rgb]{0.25,0.63,0.44}{{#1}}}
    \newcommand{\CharTok}[1]{\textcolor[rgb]{0.25,0.44,0.63}{{#1}}}
    \newcommand{\StringTok}[1]{\textcolor[rgb]{0.25,0.44,0.63}{{#1}}}
    \newcommand{\CommentTok}[1]{\textcolor[rgb]{0.38,0.63,0.69}{\textit{{#1}}}}
    \newcommand{\OtherTok}[1]{\textcolor[rgb]{0.00,0.44,0.13}{{#1}}}
    \newcommand{\AlertTok}[1]{\textcolor[rgb]{1.00,0.00,0.00}{\textbf{{#1}}}}
    \newcommand{\FunctionTok}[1]{\textcolor[rgb]{0.02,0.16,0.49}{{#1}}}
    \newcommand{\RegionMarkerTok}[1]{{#1}}
    \newcommand{\ErrorTok}[1]{\textcolor[rgb]{1.00,0.00,0.00}{\textbf{{#1}}}}
    \newcommand{\NormalTok}[1]{{#1}}
    
    % Additional commands for more recent versions of Pandoc
    \newcommand{\ConstantTok}[1]{\textcolor[rgb]{0.53,0.00,0.00}{{#1}}}
    \newcommand{\SpecialCharTok}[1]{\textcolor[rgb]{0.25,0.44,0.63}{{#1}}}
    \newcommand{\VerbatimStringTok}[1]{\textcolor[rgb]{0.25,0.44,0.63}{{#1}}}
    \newcommand{\SpecialStringTok}[1]{\textcolor[rgb]{0.73,0.40,0.53}{{#1}}}
    \newcommand{\ImportTok}[1]{{#1}}
    \newcommand{\DocumentationTok}[1]{\textcolor[rgb]{0.73,0.13,0.13}{\textit{{#1}}}}
    \newcommand{\AnnotationTok}[1]{\textcolor[rgb]{0.38,0.63,0.69}{\textbf{\textit{{#1}}}}}
    \newcommand{\CommentVarTok}[1]{\textcolor[rgb]{0.38,0.63,0.69}{\textbf{\textit{{#1}}}}}
    \newcommand{\VariableTok}[1]{\textcolor[rgb]{0.10,0.09,0.49}{{#1}}}
    \newcommand{\ControlFlowTok}[1]{\textcolor[rgb]{0.00,0.44,0.13}{\textbf{{#1}}}}
    \newcommand{\OperatorTok}[1]{\textcolor[rgb]{0.40,0.40,0.40}{{#1}}}
    \newcommand{\BuiltInTok}[1]{{#1}}
    \newcommand{\ExtensionTok}[1]{{#1}}
    \newcommand{\PreprocessorTok}[1]{\textcolor[rgb]{0.74,0.48,0.00}{{#1}}}
    \newcommand{\AttributeTok}[1]{\textcolor[rgb]{0.49,0.56,0.16}{{#1}}}
    \newcommand{\InformationTok}[1]{\textcolor[rgb]{0.38,0.63,0.69}{\textbf{\textit{{#1}}}}}
    \newcommand{\WarningTok}[1]{\textcolor[rgb]{0.38,0.63,0.69}{\textbf{\textit{{#1}}}}}
    
    
    % Define a nice break command that doesn't care if a line doesn't already
    % exist.
    \def\br{\hspace*{\fill} \\* }
    % Math Jax compatability definitions
    \def\gt{>}
    \def\lt{<}
    % Document parameters
    \title{Untitled3}
    
    
    

    % Pygments definitions
    
\makeatletter
\def\PY@reset{\let\PY@it=\relax \let\PY@bf=\relax%
    \let\PY@ul=\relax \let\PY@tc=\relax%
    \let\PY@bc=\relax \let\PY@ff=\relax}
\def\PY@tok#1{\csname PY@tok@#1\endcsname}
\def\PY@toks#1+{\ifx\relax#1\empty\else%
    \PY@tok{#1}\expandafter\PY@toks\fi}
\def\PY@do#1{\PY@bc{\PY@tc{\PY@ul{%
    \PY@it{\PY@bf{\PY@ff{#1}}}}}}}
\def\PY#1#2{\PY@reset\PY@toks#1+\relax+\PY@do{#2}}

\expandafter\def\csname PY@tok@w\endcsname{\def\PY@tc##1{\textcolor[rgb]{0.73,0.73,0.73}{##1}}}
\expandafter\def\csname PY@tok@c\endcsname{\let\PY@it=\textit\def\PY@tc##1{\textcolor[rgb]{0.25,0.50,0.50}{##1}}}
\expandafter\def\csname PY@tok@cp\endcsname{\def\PY@tc##1{\textcolor[rgb]{0.74,0.48,0.00}{##1}}}
\expandafter\def\csname PY@tok@k\endcsname{\let\PY@bf=\textbf\def\PY@tc##1{\textcolor[rgb]{0.00,0.50,0.00}{##1}}}
\expandafter\def\csname PY@tok@kp\endcsname{\def\PY@tc##1{\textcolor[rgb]{0.00,0.50,0.00}{##1}}}
\expandafter\def\csname PY@tok@kt\endcsname{\def\PY@tc##1{\textcolor[rgb]{0.69,0.00,0.25}{##1}}}
\expandafter\def\csname PY@tok@o\endcsname{\def\PY@tc##1{\textcolor[rgb]{0.40,0.40,0.40}{##1}}}
\expandafter\def\csname PY@tok@ow\endcsname{\let\PY@bf=\textbf\def\PY@tc##1{\textcolor[rgb]{0.67,0.13,1.00}{##1}}}
\expandafter\def\csname PY@tok@nb\endcsname{\def\PY@tc##1{\textcolor[rgb]{0.00,0.50,0.00}{##1}}}
\expandafter\def\csname PY@tok@nf\endcsname{\def\PY@tc##1{\textcolor[rgb]{0.00,0.00,1.00}{##1}}}
\expandafter\def\csname PY@tok@nc\endcsname{\let\PY@bf=\textbf\def\PY@tc##1{\textcolor[rgb]{0.00,0.00,1.00}{##1}}}
\expandafter\def\csname PY@tok@nn\endcsname{\let\PY@bf=\textbf\def\PY@tc##1{\textcolor[rgb]{0.00,0.00,1.00}{##1}}}
\expandafter\def\csname PY@tok@ne\endcsname{\let\PY@bf=\textbf\def\PY@tc##1{\textcolor[rgb]{0.82,0.25,0.23}{##1}}}
\expandafter\def\csname PY@tok@nv\endcsname{\def\PY@tc##1{\textcolor[rgb]{0.10,0.09,0.49}{##1}}}
\expandafter\def\csname PY@tok@no\endcsname{\def\PY@tc##1{\textcolor[rgb]{0.53,0.00,0.00}{##1}}}
\expandafter\def\csname PY@tok@nl\endcsname{\def\PY@tc##1{\textcolor[rgb]{0.63,0.63,0.00}{##1}}}
\expandafter\def\csname PY@tok@ni\endcsname{\let\PY@bf=\textbf\def\PY@tc##1{\textcolor[rgb]{0.60,0.60,0.60}{##1}}}
\expandafter\def\csname PY@tok@na\endcsname{\def\PY@tc##1{\textcolor[rgb]{0.49,0.56,0.16}{##1}}}
\expandafter\def\csname PY@tok@nt\endcsname{\let\PY@bf=\textbf\def\PY@tc##1{\textcolor[rgb]{0.00,0.50,0.00}{##1}}}
\expandafter\def\csname PY@tok@nd\endcsname{\def\PY@tc##1{\textcolor[rgb]{0.67,0.13,1.00}{##1}}}
\expandafter\def\csname PY@tok@s\endcsname{\def\PY@tc##1{\textcolor[rgb]{0.73,0.13,0.13}{##1}}}
\expandafter\def\csname PY@tok@sd\endcsname{\let\PY@it=\textit\def\PY@tc##1{\textcolor[rgb]{0.73,0.13,0.13}{##1}}}
\expandafter\def\csname PY@tok@si\endcsname{\let\PY@bf=\textbf\def\PY@tc##1{\textcolor[rgb]{0.73,0.40,0.53}{##1}}}
\expandafter\def\csname PY@tok@se\endcsname{\let\PY@bf=\textbf\def\PY@tc##1{\textcolor[rgb]{0.73,0.40,0.13}{##1}}}
\expandafter\def\csname PY@tok@sr\endcsname{\def\PY@tc##1{\textcolor[rgb]{0.73,0.40,0.53}{##1}}}
\expandafter\def\csname PY@tok@ss\endcsname{\def\PY@tc##1{\textcolor[rgb]{0.10,0.09,0.49}{##1}}}
\expandafter\def\csname PY@tok@sx\endcsname{\def\PY@tc##1{\textcolor[rgb]{0.00,0.50,0.00}{##1}}}
\expandafter\def\csname PY@tok@m\endcsname{\def\PY@tc##1{\textcolor[rgb]{0.40,0.40,0.40}{##1}}}
\expandafter\def\csname PY@tok@gh\endcsname{\let\PY@bf=\textbf\def\PY@tc##1{\textcolor[rgb]{0.00,0.00,0.50}{##1}}}
\expandafter\def\csname PY@tok@gu\endcsname{\let\PY@bf=\textbf\def\PY@tc##1{\textcolor[rgb]{0.50,0.00,0.50}{##1}}}
\expandafter\def\csname PY@tok@gd\endcsname{\def\PY@tc##1{\textcolor[rgb]{0.63,0.00,0.00}{##1}}}
\expandafter\def\csname PY@tok@gi\endcsname{\def\PY@tc##1{\textcolor[rgb]{0.00,0.63,0.00}{##1}}}
\expandafter\def\csname PY@tok@gr\endcsname{\def\PY@tc##1{\textcolor[rgb]{1.00,0.00,0.00}{##1}}}
\expandafter\def\csname PY@tok@ge\endcsname{\let\PY@it=\textit}
\expandafter\def\csname PY@tok@gs\endcsname{\let\PY@bf=\textbf}
\expandafter\def\csname PY@tok@gp\endcsname{\let\PY@bf=\textbf\def\PY@tc##1{\textcolor[rgb]{0.00,0.00,0.50}{##1}}}
\expandafter\def\csname PY@tok@go\endcsname{\def\PY@tc##1{\textcolor[rgb]{0.53,0.53,0.53}{##1}}}
\expandafter\def\csname PY@tok@gt\endcsname{\def\PY@tc##1{\textcolor[rgb]{0.00,0.27,0.87}{##1}}}
\expandafter\def\csname PY@tok@err\endcsname{\def\PY@bc##1{\setlength{\fboxsep}{0pt}\fcolorbox[rgb]{1.00,0.00,0.00}{1,1,1}{\strut ##1}}}
\expandafter\def\csname PY@tok@kc\endcsname{\let\PY@bf=\textbf\def\PY@tc##1{\textcolor[rgb]{0.00,0.50,0.00}{##1}}}
\expandafter\def\csname PY@tok@kd\endcsname{\let\PY@bf=\textbf\def\PY@tc##1{\textcolor[rgb]{0.00,0.50,0.00}{##1}}}
\expandafter\def\csname PY@tok@kn\endcsname{\let\PY@bf=\textbf\def\PY@tc##1{\textcolor[rgb]{0.00,0.50,0.00}{##1}}}
\expandafter\def\csname PY@tok@kr\endcsname{\let\PY@bf=\textbf\def\PY@tc##1{\textcolor[rgb]{0.00,0.50,0.00}{##1}}}
\expandafter\def\csname PY@tok@bp\endcsname{\def\PY@tc##1{\textcolor[rgb]{0.00,0.50,0.00}{##1}}}
\expandafter\def\csname PY@tok@fm\endcsname{\def\PY@tc##1{\textcolor[rgb]{0.00,0.00,1.00}{##1}}}
\expandafter\def\csname PY@tok@vc\endcsname{\def\PY@tc##1{\textcolor[rgb]{0.10,0.09,0.49}{##1}}}
\expandafter\def\csname PY@tok@vg\endcsname{\def\PY@tc##1{\textcolor[rgb]{0.10,0.09,0.49}{##1}}}
\expandafter\def\csname PY@tok@vi\endcsname{\def\PY@tc##1{\textcolor[rgb]{0.10,0.09,0.49}{##1}}}
\expandafter\def\csname PY@tok@vm\endcsname{\def\PY@tc##1{\textcolor[rgb]{0.10,0.09,0.49}{##1}}}
\expandafter\def\csname PY@tok@sa\endcsname{\def\PY@tc##1{\textcolor[rgb]{0.73,0.13,0.13}{##1}}}
\expandafter\def\csname PY@tok@sb\endcsname{\def\PY@tc##1{\textcolor[rgb]{0.73,0.13,0.13}{##1}}}
\expandafter\def\csname PY@tok@sc\endcsname{\def\PY@tc##1{\textcolor[rgb]{0.73,0.13,0.13}{##1}}}
\expandafter\def\csname PY@tok@dl\endcsname{\def\PY@tc##1{\textcolor[rgb]{0.73,0.13,0.13}{##1}}}
\expandafter\def\csname PY@tok@s2\endcsname{\def\PY@tc##1{\textcolor[rgb]{0.73,0.13,0.13}{##1}}}
\expandafter\def\csname PY@tok@sh\endcsname{\def\PY@tc##1{\textcolor[rgb]{0.73,0.13,0.13}{##1}}}
\expandafter\def\csname PY@tok@s1\endcsname{\def\PY@tc##1{\textcolor[rgb]{0.73,0.13,0.13}{##1}}}
\expandafter\def\csname PY@tok@mb\endcsname{\def\PY@tc##1{\textcolor[rgb]{0.40,0.40,0.40}{##1}}}
\expandafter\def\csname PY@tok@mf\endcsname{\def\PY@tc##1{\textcolor[rgb]{0.40,0.40,0.40}{##1}}}
\expandafter\def\csname PY@tok@mh\endcsname{\def\PY@tc##1{\textcolor[rgb]{0.40,0.40,0.40}{##1}}}
\expandafter\def\csname PY@tok@mi\endcsname{\def\PY@tc##1{\textcolor[rgb]{0.40,0.40,0.40}{##1}}}
\expandafter\def\csname PY@tok@il\endcsname{\def\PY@tc##1{\textcolor[rgb]{0.40,0.40,0.40}{##1}}}
\expandafter\def\csname PY@tok@mo\endcsname{\def\PY@tc##1{\textcolor[rgb]{0.40,0.40,0.40}{##1}}}
\expandafter\def\csname PY@tok@ch\endcsname{\let\PY@it=\textit\def\PY@tc##1{\textcolor[rgb]{0.25,0.50,0.50}{##1}}}
\expandafter\def\csname PY@tok@cm\endcsname{\let\PY@it=\textit\def\PY@tc##1{\textcolor[rgb]{0.25,0.50,0.50}{##1}}}
\expandafter\def\csname PY@tok@cpf\endcsname{\let\PY@it=\textit\def\PY@tc##1{\textcolor[rgb]{0.25,0.50,0.50}{##1}}}
\expandafter\def\csname PY@tok@c1\endcsname{\let\PY@it=\textit\def\PY@tc##1{\textcolor[rgb]{0.25,0.50,0.50}{##1}}}
\expandafter\def\csname PY@tok@cs\endcsname{\let\PY@it=\textit\def\PY@tc##1{\textcolor[rgb]{0.25,0.50,0.50}{##1}}}

\def\PYZbs{\char`\\}
\def\PYZus{\char`\_}
\def\PYZob{\char`\{}
\def\PYZcb{\char`\}}
\def\PYZca{\char`\^}
\def\PYZam{\char`\&}
\def\PYZlt{\char`\<}
\def\PYZgt{\char`\>}
\def\PYZsh{\char`\#}
\def\PYZpc{\char`\%}
\def\PYZdl{\char`\$}
\def\PYZhy{\char`\-}
\def\PYZsq{\char`\'}
\def\PYZdq{\char`\"}
\def\PYZti{\char`\~}
% for compatibility with earlier versions
\def\PYZat{@}
\def\PYZlb{[}
\def\PYZrb{]}
\makeatother


    % Exact colors from NB
    \definecolor{incolor}{rgb}{0.0, 0.0, 0.5}
    \definecolor{outcolor}{rgb}{0.545, 0.0, 0.0}



    
    % Prevent overflowing lines due to hard-to-break entities
    \sloppy 
    % Setup hyperref package
    \hypersetup{
      breaklinks=true,  % so long urls are correctly broken across lines
      colorlinks=true,
      urlcolor=urlcolor,
      linkcolor=linkcolor,
      citecolor=citecolor,
      }
    % Slightly bigger margins than the latex defaults
    
    \geometry{verbose,tmargin=1in,bmargin=1in,lmargin=1in,rmargin=1in}
    
    

    \begin{document}
    
    
    \maketitle
    
    

    
    \section{使用MCMC中的Metropolis-Hasting方法进行随机抽样}\label{ux4f7fux7528mcmcux4e2dux7684metropolis-hastingux65b9ux6cd5ux8fdbux884cux968fux673aux62bdux6837}

    \subsection{一.马尔科夫链的一般结构}\label{ux4e00.ux9a6cux5c14ux79d1ux592bux94feux7684ux4e00ux822cux7ed3ux6784}

\subsubsection{\texorpdfstring{马尔科夫链是一条状态链,可表示为\(\{\pi(1),\pi(2)\cdots \pi(i),\pi(j)\cdots\}\),结合概率论的内容,可将某概率密度函数\(F\)表示为如下形式:\[\lim_{n\to+\infty}\{\cdots ,F(-\frac{2}{n}),F(-\frac{1}{n}),F(0),F(\frac{1}{n}),F(\frac{2}{n}),\cdots\}\]且满足\[\int_{-\infty}^{+\infty} F(x)dx=1\]}{马尔科夫链是一条状态链,可表示为\textbackslash{}\{\textbackslash{}pi(1),\textbackslash{}pi(2)\textbackslash{}cdots \textbackslash{}pi(i),\textbackslash{}pi(j)\textbackslash{}cdots\textbackslash{}\},结合概率论的内容,可将某概率密度函数F表示为如下形式:\textbackslash{}lim\_\{n\textbackslash{}to+\textbackslash{}infty\}\textbackslash{}\{\textbackslash{}cdots ,F(-\textbackslash{}frac\{2\}\{n\}),F(-\textbackslash{}frac\{1\}\{n\}),F(0),F(\textbackslash{}frac\{1\}\{n\}),F(\textbackslash{}frac\{2\}\{n\}),\textbackslash{}cdots\textbackslash{}\}且满足\textbackslash{}int\_\{-\textbackslash{}infty\}\^{}\{+\textbackslash{}infty\} F(x)dx=1}}\label{ux9a6cux5c14ux79d1ux592bux94feux662fux4e00ux6761ux72b6ux6001ux94feux53efux8868ux793aux4e3api1pi2cdots-piipijcdotsux7ed3ux5408ux6982ux7387ux8bbaux7684ux5185ux5bb9ux53efux5c06ux67d0ux6982ux7387ux5bc6ux5ea6ux51fdux6570fux8868ux793aux4e3aux5982ux4e0bux5f62ux5f0flim_ntoinftycdots-f-frac2nf-frac1nf0ffrac1nffrac2ncdotsux4e14ux6ee1ux8db3int_-inftyinfty-fxdx1}

\subsubsection{\texorpdfstring{状态链的转移通过转移概率矩阵\(P=\{p_{ij}\}\)实现,\(P\)的结构如下:}{状态链的转移通过转移概率矩阵P=\textbackslash{}\{p\_\{ij\}\textbackslash{}\}实现,P的结构如下:}}\label{ux72b6ux6001ux94feux7684ux8f6cux79fbux901aux8fc7ux8f6cux79fbux6982ux7387ux77e9ux9635pp_ijux5b9eux73b0pux7684ux7ed3ux6784ux5982ux4e0b}

\subsubsection{\texorpdfstring{\[
        \left (\begin{matrix}
        p_{11} & p_{12} & \cdots &p_{1j}& \cdots  \\
        p_{21}  & p_{22} & \cdots &p_{2j}& \cdots \\
        \vdots & \vdots & \ddots & \vdots & \cdots \\
        p_{i1} & p_{i2} & \cdots & p_{ij} & \cdots \\
        \vdots & \vdots & \vdots & \vdots & \ddots \\
        \end{matrix} \right)
  \]}{
        \textbackslash{}left (\textbackslash{}begin\{matrix\}
        p\_\{11\} \& p\_\{12\} \& \textbackslash{}cdots \&p\_\{1j\}\& \textbackslash{}cdots  \textbackslash{}\textbackslash{}
        p\_\{21\}  \& p\_\{22\} \& \textbackslash{}cdots \&p\_\{2j\}\& \textbackslash{}cdots \textbackslash{}\textbackslash{}
        \textbackslash{}vdots \& \textbackslash{}vdots \& \textbackslash{}ddots \& \textbackslash{}vdots \& \textbackslash{}cdots \textbackslash{}\textbackslash{}
        p\_\{i1\} \& p\_\{i2\} \& \textbackslash{}cdots \& p\_\{ij\} \& \textbackslash{}cdots \textbackslash{}\textbackslash{}
        \textbackslash{}vdots \& \textbackslash{}vdots \& \textbackslash{}vdots \& \textbackslash{}vdots \& \textbackslash{}ddots \textbackslash{}\textbackslash{}
        \textbackslash{}end\{matrix\} \textbackslash{}right)
  }}\label{left-beginmatrix-p_11-p_12-cdots-p_1j-cdots-p_21-p_22-cdots-p_2j-cdots-vdots-vdots-ddots-vdots-cdots-p_i1-p_i2-cdots-p_ij-cdots-vdots-vdots-vdots-vdots-ddots-endmatrix-right}

\subsubsection{\texorpdfstring{其中每行元素满足和式\[\sum_{j=1}^{\infty} p_{ij}=1\]}{其中每行元素满足和式\textbackslash{}sum\_\{j=1\}\^{}\{\textbackslash{}infty\} p\_\{ij\}=1}}\label{ux5176ux4e2dux6bcfux884cux5143ux7d20ux6ee1ux8db3ux548cux5f0fsum_j1infty-p_ij1}

\subsubsection{\texorpdfstring{转移概率矩阵若不为n次幂等矩阵,则称具有该转移概率矩阵的马尔科夫过程具有遍历性,具有遍历性的马尔科夫过程,满足性质\[\lim_{n\to+\infty}p_{ij}^n=\bar\pi(j)\]可知\(P^n\)中元素只与列有关,格式如下:}{转移概率矩阵若不为n次幂等矩阵,则称具有该转移概率矩阵的马尔科夫过程具有遍历性,具有遍历性的马尔科夫过程,满足性质\textbackslash{}lim\_\{n\textbackslash{}to+\textbackslash{}infty\}p\_\{ij\}\^{}n=\textbackslash{}bar\textbackslash{}pi(j)可知P\^{}n中元素只与列有关,格式如下:}}\label{ux8f6cux79fbux6982ux7387ux77e9ux9635ux82e5ux4e0dux4e3anux6b21ux5e42ux7b49ux77e9ux9635ux5219ux79f0ux5177ux6709ux8be5ux8f6cux79fbux6982ux7387ux77e9ux9635ux7684ux9a6cux5c14ux79d1ux592bux8fc7ux7a0bux5177ux6709ux904dux5386ux6027ux5177ux6709ux904dux5386ux6027ux7684ux9a6cux5c14ux79d1ux592bux8fc7ux7a0bux6ee1ux8db3ux6027ux8d28lim_ntoinftyp_ijnbarpijux53efux77e5pnux4e2dux5143ux7d20ux53eaux4e0eux5217ux6709ux5173ux683cux5f0fux5982ux4e0b}

\subsubsection{\texorpdfstring{\[ P^n=
        \left (\begin{matrix}
        \bar\pi(1) & \bar\pi(2) & \cdots &\bar\pi(j)& \cdots  \\
        \bar\pi(1)  & \bar\pi(2) & \cdots &\bar\pi(j)& \cdots \\
        \vdots & \vdots & \ddots & \vdots & \cdots \\
       \bar\pi(1) & \bar\pi(2) & \cdots & \bar\pi(j) & \cdots \\
        \vdots & \vdots & \vdots & \vdots & \ddots \\
        \end{matrix} \right)\]}{ P\^{}n=
        \textbackslash{}left (\textbackslash{}begin\{matrix\}
        \textbackslash{}bar\textbackslash{}pi(1) \& \textbackslash{}bar\textbackslash{}pi(2) \& \textbackslash{}cdots \&\textbackslash{}bar\textbackslash{}pi(j)\& \textbackslash{}cdots  \textbackslash{}\textbackslash{}
        \textbackslash{}bar\textbackslash{}pi(1)  \& \textbackslash{}bar\textbackslash{}pi(2) \& \textbackslash{}cdots \&\textbackslash{}bar\textbackslash{}pi(j)\& \textbackslash{}cdots \textbackslash{}\textbackslash{}
        \textbackslash{}vdots \& \textbackslash{}vdots \& \textbackslash{}ddots \& \textbackslash{}vdots \& \textbackslash{}cdots \textbackslash{}\textbackslash{}
       \textbackslash{}bar\textbackslash{}pi(1) \& \textbackslash{}bar\textbackslash{}pi(2) \& \textbackslash{}cdots \& \textbackslash{}bar\textbackslash{}pi(j) \& \textbackslash{}cdots \textbackslash{}\textbackslash{}
        \textbackslash{}vdots \& \textbackslash{}vdots \& \textbackslash{}vdots \& \textbackslash{}vdots \& \textbackslash{}ddots \textbackslash{}\textbackslash{}
        \textbackslash{}end\{matrix\} \textbackslash{}right)}}\label{pn-left-beginmatrix-barpi1-barpi2-cdots-barpij-cdots-barpi1-barpi2-cdots-barpij-cdots-vdots-vdots-ddots-vdots-cdots-barpi1-barpi2-cdots-barpij-cdots-vdots-vdots-vdots-vdots-ddots-endmatrix-right}

\subsubsection{\texorpdfstring{证明:由n次幂矩阵的计算方法,作矩阵分解得\(P=A\Lambda A^{-1}\),其中\(\Lambda\)为对角矩阵,形式如下:}{证明:由n次幂矩阵的计算方法,作矩阵分解得P=A\textbackslash{}Lambda A\^{}\{-1\},其中\textbackslash{}Lambda为对角矩阵,形式如下:}}\label{ux8bc1ux660eux7531nux6b21ux5e42ux77e9ux9635ux7684ux8ba1ux7b97ux65b9ux6cd5ux4f5cux77e9ux9635ux5206ux89e3ux5f97palambda-a-1ux5176ux4e2dlambdaux4e3aux5bf9ux89d2ux77e9ux9635ux5f62ux5f0fux5982ux4e0b}

\subsubsection{\texorpdfstring{\[ \Lambda=
        \left (\begin{matrix}
        \lambda_1 & 0 & \cdots& 0& \cdots  \\
        0  & \lambda_2 & \cdots & 0 & \cdots \\
        \vdots & \vdots & \ddots & \vdots & \cdots \\
       0 & 0 & \cdots & \lambda_j & \cdots \\
        \vdots & \vdots & \vdots & \vdots & \ddots \\
        \end{matrix} \right)\]}{ \textbackslash{}Lambda=
        \textbackslash{}left (\textbackslash{}begin\{matrix\}
        \textbackslash{}lambda\_1 \& 0 \& \textbackslash{}cdots\& 0\& \textbackslash{}cdots  \textbackslash{}\textbackslash{}
        0  \& \textbackslash{}lambda\_2 \& \textbackslash{}cdots \& 0 \& \textbackslash{}cdots \textbackslash{}\textbackslash{}
        \textbackslash{}vdots \& \textbackslash{}vdots \& \textbackslash{}ddots \& \textbackslash{}vdots \& \textbackslash{}cdots \textbackslash{}\textbackslash{}
       0 \& 0 \& \textbackslash{}cdots \& \textbackslash{}lambda\_j \& \textbackslash{}cdots \textbackslash{}\textbackslash{}
        \textbackslash{}vdots \& \textbackslash{}vdots \& \textbackslash{}vdots \& \textbackslash{}vdots \& \textbackslash{}ddots \textbackslash{}\textbackslash{}
        \textbackslash{}end\{matrix\} \textbackslash{}right)}}\label{lambda-left-beginmatrix-lambda_1-0-cdots-0-cdots-0-lambda_2-cdots-0-cdots-vdots-vdots-ddots-vdots-cdots-0-0-cdots-lambda_j-cdots-vdots-vdots-vdots-vdots-ddots-endmatrix-right}

\subsubsection{\texorpdfstring{根据该分解易知\(P^n=A\Lambda^nA^{-1}\)}{根据该分解易知P\^{}n=A\textbackslash{}Lambda\^{}nA\^{}\{-1\}}}\label{ux6839ux636eux8be5ux5206ux89e3ux6613ux77e5pnalambdana-1}

\subsubsection{\texorpdfstring{\[P=A\Lambda A^{-1}\Leftrightarrow 
     PA=A\Lambda
     \]}{P=A\textbackslash{}Lambda A\^{}\{-1\}\textbackslash{}Leftrightarrow 
     PA=A\textbackslash{}Lambda
     }}\label{palambda-a-1leftrightarrow-paalambda}

\subsubsection{\texorpdfstring{\(A\)可以写作列向量组合\(\begin{bmatrix} A^1 & A^2 & \cdots & A^j & \cdots\\ \end{bmatrix}\)}{A可以写作列向量组合\textbackslash{}begin\{bmatrix\} A\^{}1 \& A\^{}2 \& \textbackslash{}cdots \& A\^{}j \& \textbackslash{}cdots\textbackslash{}\textbackslash{} \textbackslash{}end\{bmatrix\}}}\label{aux53efux4ee5ux5199ux4f5cux5217ux5411ux91cfux7ec4ux5408beginbmatrix-a1-a2-cdots-aj-cdots-endbmatrix}

\subsubsection{\texorpdfstring{写成方程组形式则为\[\begin{split}  PA^i=\lambda_iA^i \quad (i=1,2,\cdots,j\cdots)\\
                     (P-\lambda_i I)A^i=0 \quad (i=1,2,\cdots,j\cdots)\end{split}\]}{写成方程组形式则为\textbackslash{}begin\{split\}  PA\^{}i=\textbackslash{}lambda\_iA\^{}i \textbackslash{}quad (i=1,2,\textbackslash{}cdots,j\textbackslash{}cdots)\textbackslash{}\textbackslash{}
                     (P-\textbackslash{}lambda\_i I)A\^{}i=0 \textbackslash{}quad (i=1,2,\textbackslash{}cdots,j\textbackslash{}cdots)\textbackslash{}end\{split\}}}\label{ux5199ux6210ux65b9ux7a0bux7ec4ux5f62ux5f0fux5219ux4e3abeginsplit-pailambda_iai-quad-i12cdotsjcdots-p-lambda_i-iai0-quad-i12cdotsjcdotsendsplit}

\subsubsection{\texorpdfstring{则将求解矩阵\(A\)的问题转化为求解系数矩阵为\(P-\lambda_i I\)的齐次线性方程组}{则将求解矩阵A的问题转化为求解系数矩阵为P-\textbackslash{}lambda\_i I的齐次线性方程组}}\label{ux5219ux5c06ux6c42ux89e3ux77e9ux9635aux7684ux95eeux9898ux8f6cux5316ux4e3aux6c42ux89e3ux7cfbux6570ux77e9ux9635ux4e3ap-lambda_i-iux7684ux9f50ux6b21ux7ebfux6027ux65b9ux7a0bux7ec4}

\subsubsection{\texorpdfstring{转移矩阵若为n次幂等矩阵,即满足:\[\Lambda^n =\Lambda \Leftrightarrow \lambda_i^n =\lambda_i\quad (i=1,2,\cdots,j\cdots)\\
\Updownarrow\\
\lambda_i\in\{0\}\cup\{\cos(\frac{2k\pi}{n-1} )+i\sin(\frac{2k\pi}{n-1} )|k=1,2,\cdots,n-1\}\quad (i=1,2,\cdots,j\cdots)\]}{转移矩阵若为n次幂等矩阵,即满足:\textbackslash{}Lambda\^{}n =\textbackslash{}Lambda \textbackslash{}Leftrightarrow \textbackslash{}lambda\_i\^{}n =\textbackslash{}lambda\_i\textbackslash{}quad (i=1,2,\textbackslash{}cdots,j\textbackslash{}cdots)\textbackslash{}\textbackslash{}
\textbackslash{}Updownarrow\textbackslash{}\textbackslash{}
\textbackslash{}lambda\_i\textbackslash{}in\textbackslash{}\{0\textbackslash{}\}\textbackslash{}cup\textbackslash{}\{\textbackslash{}cos(\textbackslash{}frac\{2k\textbackslash{}pi\}\{n-1\} )+i\textbackslash{}sin(\textbackslash{}frac\{2k\textbackslash{}pi\}\{n-1\} )\textbar{}k=1,2,\textbackslash{}cdots,n-1\textbackslash{}\}\textbackslash{}quad (i=1,2,\textbackslash{}cdots,j\textbackslash{}cdots)}}\label{ux8f6cux79fbux77e9ux9635ux82e5ux4e3anux6b21ux5e42ux7b49ux77e9ux9635ux5373ux6ee1ux8db3lambdan-lambda-leftrightarrow-lambda_in-lambda_iquad-i12cdotsjcdotsupdownarrowlambda_iin0cupcosfrac2kpin-1-isinfrac2kpin-1-k12cdotsn-1quad-i12cdotsjcdots}

\subsubsection{(对于此处特征值的代数重数大于特征向量的代数重数的情况,引入Jordan标准型矩阵)}\label{ux5bf9ux4e8eux6b64ux5904ux7279ux5f81ux503cux7684ux4ee3ux6570ux91cdux6570ux5927ux4e8eux7279ux5f81ux5411ux91cfux7684ux4ee3ux6570ux91cdux6570ux7684ux60c5ux51b5ux5f15ux5165jordanux6807ux51c6ux578bux77e9ux9635}

\subsubsection{此时该马尔科夫过程不具有遍历性}\label{ux6b64ux65f6ux8be5ux9a6cux5c14ux79d1ux592bux8fc7ux7a0bux4e0dux5177ux6709ux904dux5386ux6027}

    \subsection{二.稳定状态链的基本概念}\label{ux4e8c.ux7a33ux5b9aux72b6ux6001ux94feux7684ux57faux672cux6982ux5ff5}

\subsubsection{由马尔科夫过程具有极限状态,可知马尔科夫链的稳态条件满足:}\label{ux7531ux9a6cux5c14ux79d1ux592bux8fc7ux7a0bux5177ux6709ux6781ux9650ux72b6ux6001ux53efux77e5ux9a6cux5c14ux79d1ux592bux94feux7684ux7a33ux6001ux6761ux4ef6ux6ee1ux8db3}

\subsubsection{\texorpdfstring{\[ 
         \left (\begin{matrix}
        \bar\pi(1)\\ \bar\pi(2) \\ \vdots \\ \bar\pi(j)\\ \vdots  \\
        \end{matrix} \right)^{-1}
        \left (\begin{matrix}
        p_{11} & p_{12} & \cdots &p_{1j}& \cdots  \\
        p_{21}  & p_{22} & \cdots &p_{2j}& \cdots \\
        \vdots & \vdots & \ddots & \vdots & \cdots \\
        p_{i1} & p_{i2} & \cdots & p_{ij} & \cdots \\
        \vdots & \vdots & \vdots & \vdots & \ddots \\
        \end{matrix} \right)=
         \left (\begin{matrix}
        \bar\pi(1)\\ \bar\pi(2) \\ \vdots \\ \bar\pi(j)\\ \vdots  \\
        \end{matrix} \right)^{-1}
        \]}{ 
         \textbackslash{}left (\textbackslash{}begin\{matrix\}
        \textbackslash{}bar\textbackslash{}pi(1)\textbackslash{}\textbackslash{} \textbackslash{}bar\textbackslash{}pi(2) \textbackslash{}\textbackslash{} \textbackslash{}vdots \textbackslash{}\textbackslash{} \textbackslash{}bar\textbackslash{}pi(j)\textbackslash{}\textbackslash{} \textbackslash{}vdots  \textbackslash{}\textbackslash{}
        \textbackslash{}end\{matrix\} \textbackslash{}right)\^{}\{-1\}
        \textbackslash{}left (\textbackslash{}begin\{matrix\}
        p\_\{11\} \& p\_\{12\} \& \textbackslash{}cdots \&p\_\{1j\}\& \textbackslash{}cdots  \textbackslash{}\textbackslash{}
        p\_\{21\}  \& p\_\{22\} \& \textbackslash{}cdots \&p\_\{2j\}\& \textbackslash{}cdots \textbackslash{}\textbackslash{}
        \textbackslash{}vdots \& \textbackslash{}vdots \& \textbackslash{}ddots \& \textbackslash{}vdots \& \textbackslash{}cdots \textbackslash{}\textbackslash{}
        p\_\{i1\} \& p\_\{i2\} \& \textbackslash{}cdots \& p\_\{ij\} \& \textbackslash{}cdots \textbackslash{}\textbackslash{}
        \textbackslash{}vdots \& \textbackslash{}vdots \& \textbackslash{}vdots \& \textbackslash{}vdots \& \textbackslash{}ddots \textbackslash{}\textbackslash{}
        \textbackslash{}end\{matrix\} \textbackslash{}right)=
         \textbackslash{}left (\textbackslash{}begin\{matrix\}
        \textbackslash{}bar\textbackslash{}pi(1)\textbackslash{}\textbackslash{} \textbackslash{}bar\textbackslash{}pi(2) \textbackslash{}\textbackslash{} \textbackslash{}vdots \textbackslash{}\textbackslash{} \textbackslash{}bar\textbackslash{}pi(j)\textbackslash{}\textbackslash{} \textbackslash{}vdots  \textbackslash{}\textbackslash{}
        \textbackslash{}end\{matrix\} \textbackslash{}right)\^{}\{-1\}
        }}\label{left-beginmatrix-barpi1-barpi2-vdots-barpij-vdots-endmatrix-right-1-left-beginmatrix-p_11-p_12-cdots-p_1j-cdots-p_21-p_22-cdots-p_2j-cdots-vdots-vdots-ddots-vdots-cdots-p_i1-p_i2-cdots-p_ij-cdots-vdots-vdots-vdots-vdots-ddots-endmatrix-right-left-beginmatrix-barpi1-barpi2-vdots-barpij-vdots-endmatrix-right-1}

\subsubsection{即满足:}\label{ux5373ux6ee1ux8db3}

\subsubsection{\texorpdfstring{\[\bar\pi(j)=\sum_{i=1}^{\infty} \bar\pi(i)p_{ij}\]}{\textbackslash{}bar\textbackslash{}pi(j)=\textbackslash{}sum\_\{i=1\}\^{}\{\textbackslash{}infty\} \textbackslash{}bar\textbackslash{}pi(i)p\_\{ij\}}}\label{barpijsum_i1infty-barpiip_ij}

\subsubsection{补充以下条件,联立知:}\label{ux8865ux5145ux4ee5ux4e0bux6761ux4ef6ux8054ux7acbux77e5}

\subsubsection{\texorpdfstring{\[
\begin{cases}
\bar\pi(j)=\sum_{i=1}^{\infty} \bar\pi(i)p_{ij}\\
\quad\\
\sum_{i=1}^{\infty} p_{ji}=1
\end{cases}
\Leftrightarrow
\sum_{i=1}^{\infty} \bar\pi(j)p_{ji}=\sum_{i=1}^{\infty} \bar\pi(i)p_{ij}\]}{
\textbackslash{}begin\{cases\}
\textbackslash{}bar\textbackslash{}pi(j)=\textbackslash{}sum\_\{i=1\}\^{}\{\textbackslash{}infty\} \textbackslash{}bar\textbackslash{}pi(i)p\_\{ij\}\textbackslash{}\textbackslash{}
\textbackslash{}quad\textbackslash{}\textbackslash{}
\textbackslash{}sum\_\{i=1\}\^{}\{\textbackslash{}infty\} p\_\{ji\}=1
\textbackslash{}end\{cases\}
\textbackslash{}Leftrightarrow
\textbackslash{}sum\_\{i=1\}\^{}\{\textbackslash{}infty\} \textbackslash{}bar\textbackslash{}pi(j)p\_\{ji\}=\textbackslash{}sum\_\{i=1\}\^{}\{\textbackslash{}infty\} \textbackslash{}bar\textbackslash{}pi(i)p\_\{ij\}}}\label{begincasesbarpijsum_i1infty-barpiip_ijquadsum_i1infty-p_ji1endcasesleftrightarrowsum_i1infty-barpijp_jisum_i1infty-barpiip_ij}

\subsubsection{马尔可夫链的细致平稳(detailed
balance)是状态链达到稳定状态的充分不必要条件:}\label{ux9a6cux5c14ux53efux592bux94feux7684ux7ec6ux81f4ux5e73ux7a33detailed-balanceux662fux72b6ux6001ux94feux8fbeux5230ux7a33ux5b9aux72b6ux6001ux7684ux5145ux5206ux4e0dux5fc5ux8981ux6761ux4ef6}

\subsubsection{\texorpdfstring{\[\bar\pi(i)p_{ij}=\bar\pi(j)p_{ji}
\Rightarrow
\sum_{i=1}^{\infty} \bar\pi(j)p_{ji}=\sum_{i=1}^{\infty} \bar\pi(i)p_{ij}
\]}{\textbackslash{}bar\textbackslash{}pi(i)p\_\{ij\}=\textbackslash{}bar\textbackslash{}pi(j)p\_\{ji\}
\textbackslash{}Rightarrow
\textbackslash{}sum\_\{i=1\}\^{}\{\textbackslash{}infty\} \textbackslash{}bar\textbackslash{}pi(j)p\_\{ji\}=\textbackslash{}sum\_\{i=1\}\^{}\{\textbackslash{}infty\} \textbackslash{}bar\textbackslash{}pi(i)p\_\{ij\}
}}\label{barpiip_ijbarpijp_jirightarrowsum_i1infty-barpijp_jisum_i1infty-barpiip_ij}

\subsubsection{\texorpdfstring{达到细致平稳时,马氏链由状态\(\bar\pi(i)\)转移到状态\(\bar\pi(j)\)的概率为\(p_{ij}\),反之亦然,链的稳态条件即:两状态转移概率相等。}{达到细致平稳时,马氏链由状态\textbackslash{}bar\textbackslash{}pi(i)转移到状态\textbackslash{}bar\textbackslash{}pi(j)的概率为p\_\{ij\},反之亦然,链的稳态条件即:两状态转移概率相等。}}\label{ux8fbeux5230ux7ec6ux81f4ux5e73ux7a33ux65f6ux9a6cux6c0fux94feux7531ux72b6ux6001barpiiux8f6cux79fbux5230ux72b6ux6001barpijux7684ux6982ux7387ux4e3ap_ijux53cdux4e4bux4ea6ux7136ux94feux7684ux7a33ux6001ux6761ux4ef6ux5373ux4e24ux72b6ux6001ux8f6cux79fbux6982ux7387ux76f8ux7b49}

    \subsection{三.Metropolis Algorithm and
Exampers}\label{ux4e09.metropolis-algorithm-and-exampers}


    % Add a bibliography block to the postdoc
    
    
    
    \end{document}
